\documentclass[portrait,a4paper]{article}
\usepackage[a4paper, margin=10pt]{geometry}	
\usepackage{multicol}
\usepackage{amsmath}

% \usepackage{color}

\setlength{\columnsep}{0.5cm}
\setlength\multicolsep{0pt}


% \setlength{\columnseprule}{1pt}
% \def\columnseprulecolor{\color{black}}

%-----------------------------------------------------------------------


\begin{document}
\begin{multicols}{3}


\section*{Bilanci}
\subsection*{Sistema chiuso}
Bilancio massa \(M = \text{cost}\) \\
Bilancio energia \\
\[\sum (\dot{Q}+\dot L)_{in} = \sum (\dot{Q}+ \dot{L})_{out} + \frac{dE}{dt} \] \\
$\frac{dE}{dt} = \frac{d}{dt}\big(M \frac{w^2}{2} + Mgz + U\big)$ \\
Lavoro dilatativo, dovuto alla deformazione del contorno del sistema \\(hp: TIR) \\
Compressione: \( L_{in} = - \int P dV\) \\
Espansione: \(L_{in} = \int P dV\)

\subsection*{Sistema aperto}
Bilancio massa \(m_{in} = m_{out}\) \\
Bilancio energia \\
\[ \dot{m}e_{in} + \sum (\dot{Q}+\dot L)_{in} = \dot{m}e_{out} + \sum (\dot{Q}+ \dot{L})_{out} \] \\
Energia specifica del fluido: \\
\(e = \big(u + g\Delta z + \frac{w^2}{2} + Pv \big)\) \\
Lavoro pulsione: \(l_{puls} = Pv\) \\
Entalpia: \(h = u + l_{puls} = u + Pv\) \\
Bilancio di potenze: \\
\[ \dot{m}h_{in} + \dot{Q}_{in} = \dot{m}h_{out} + \dot{L}_{out} \] \\
Forma differenziale: \(dh + dl_{out} = dq_{in}\)
Lavoro utile in uscita: \\
\[l_{utile}^{out} = - \int_{in}^{out} vdP\]

\subsection*{Portata}
Portata massica: \( \dot m = \rho w A \) \\
Portata volumetrica: \( \dot m = \rho V \)

\section*{Secondo principio}
\subsection*{Enunciati}
\textbf{Kelvin}: É impossibile realizzare una macchina il cui unico risultato preveda che tutto il calore assorbito da una sorgente omogenea sia interamente trasformato in lavoro. \\
Macchina motrice prevede \(Q_c = L + Q_f\)
\textbf{Clausius}: É impossibile realizzare una macchina il cui unico risultato sia quello di trasferire calore da un corpo piú freddo ad uno più caldo. \\
Macchina frigorifera prevede \(Q_f + L = Q_c\)
\subsection*{Entropia}
Irreversibilitá dovute a scambio termico con \(\Delta T\) non nulli sono \textbf{esterne}. \\
Irreversibilitá dovute ad attriti, miscelamenti, turbolenza sono \textbf{interne}. \\
\textbf{Bilancio entropia}: \\
%\(S_x = \dot{m}_x\cdot s_x + \frac{\dot{Q}_x}{T}\)\\
%\[S_{in} + S_{irr} = S_{out} + \Delta S\]
\[\dot{m}_{in}s_{in} + \dot{S}_Q{_{in}} + \dot{S}_{irr} = \dot{m}_{out}s_{out} + \dot{S}_Q{_{out}} + \frac{dS}{dt}\]

\section*{Trasformazioni gas}
\subsection*{Equazioni del Tds}
Sistema chiuso: \(du + Pdv = Tds\) \\
Sistema aperto: \(dh -vdP = Tds\)
\subsection*{Gas ideali}
\(R=\frac {R*}{M_{m}}\) \\
\begin{tabular}{l|c|c}
	Monoatomica & $c_v=\frac{3}{2}R$ & $c_p=\frac{5}{2}R$ \\
	Biatomica & $c_v=\frac{5}{2}R$ & $c_p=\frac{7}{2}R$ \\
	Poliatomica & $c_v=3R$ & $c_p=4R$ \\
\end{tabular}
\textbf{Energia interna}: \(du = c_v(T_f-T_i)\) \\
\textbf{Entalpia}: \(dh = c_p(T_f-T_i)\) \\
\textbf{Entropia}: \begin{align*}
	s_2 - s_1 &= M\bigg( c_v \ln \bigg( \frac{T_2}{T_1} \bigg)
	+R \ln \bigg( \frac{v_2}{v_1} \bigg) \bigg) \\
	 &= M\bigg( c_p \ln \bigg( \frac{T_2}{T_1} \bigg)
	- R \ln \bigg( \frac{P_2}{P_1} \bigg) \bigg)
\end{align*}


\subsection*{Politropica}
Calore specifico \(c_x\) costante durante la trasformazione. \\
\textbf{Indice }della politropica: \(n = \frac{c_x - c_p}{c_x - c_v}\)
	\begin{center}\(Pv^n = \text{cost}\)\end{center}
	\begin{multicols}{2}
	\(\frac{T_f}{T_i} = (\frac{P_f}{P_i})^{\frac{n-1}{n}}\) \\
	\(\frac{T_f}{T_i} = (\frac{V_i}{V_f})^{n-1}\)\\
\end{multicols}


\subsection*{Isobara}
\begin{align*}
	q_{in} &= l_{out} + \Delta u \\
	&= \int PdV + c_v\Delta T \\
	&= P\Delta V + c_v\Delta T \\
	[G.I.] &= R\Delta T + c_v \Delta T \\
	&= c_p \Delta T \\
	q_{in}&= dh
\end{align*} 
Il lavoro uscente in un sistema aperto é nullo: \(l_{out}^{^{APERTO}}=0\)
\subsection*{Isoterma}
\textbf{Sistema chiuso}: \(\Delta u = c_v\Delta T = 0\) \\
\textbf{Sistema aperto}: \(\Delta h = c_p\Delta T = 0\)
\[ l_{out} = q_{in} = -RT\ln\frac{P_f}{P_i}\]

\subsection*{Isocora}
\(q_{in} = \Delta u\) \\
\(l_{out} = 0\)
\textbf{Sistema chiuso}: \\
\(q_{in} = l_{out} + du = \int Pdv + du = 0 + du\) \\
\textbf{Sistema aperto}:
\begin{align*}
	q_{in} &= l_{out} + dh \\
	&= -vdP + dh \\
	&= -vdP + (du + l_{puls}) \\
	&= -vdP + (du + vdP) \\
	&= du	
\end{align*}


\subsection*{Isoentropica}

\begin{center}
	\(q_{in} = 0\) \\	
	\( \gamma = \frac{c_p}{c_v} \)
\end{center}
\begin{multicols}{3}
\noindent
\[\frac{T_2}{T_1}=\frac{P_2}{P_1}^{\frac{\gamma -1}{\gamma}}\]
\[\frac{T_2}{T_1}=\frac{v_1}{v_2}^{\gamma -1}\]
\[\frac{P_2}{P_1}=\frac{v_1}{v_2}^{\gamma -1}\]
\end{multicols}

\section*{Liquidi ideali}
$ \Delta u = C\Delta T $ \\
$ \Delta h = C\Delta T + v\Delta P $ \\
$ \Delta s = C\ln\frac{T_2}{T_1} $ \\
Per i liquidi ideali una trasformazione isoentropica é anche isoterma: \\
\(\Delta s = 0 = C\ln\big(\frac{T_f}{T_i}\big) \rightarrow T_f = T_i \) \\
Per una trasformazione ISOBARA si usano le stesse leggi dei gas ideali, ma siccome il lavoro di pulsione é nullo perché \(P=\text{cost}\) abbiamo \(\Delta h = \Delta u\)
\subsection*{Miscele}
\[ X =\frac{v - v_{_{LC}}}{v_{_{VS}} - v_{_{LS}}}\]


\section*{Conduzione}
Flusso termico: \(\dot{q} = \frac{\dot{Q}}{A} \) \\
Legge Fourier che descrive flusso termico:
\[
\dot{q} = -k\frac{dT}{dx}
\]
Conducibilitá Termica: \( k = \lambda = \frac{\dot{q}L}{\Delta T}\) \\
Conservazione dell'energia
\[
\frac{d\dot{q}}{dx}=-\rho c\frac{dT}{dt}
\]
Equazione generale della Conduzione \\
\(\frac{\partial}{\partial x}\bigg( k\frac{\partial T}{\partial x} \bigg) = \rho c \frac{dT}{dt}
\) \\
\begin{tabular}{c|c}
$ R_{_{cond}}^{^{\text{lastra p.}}} =\frac{S}{KA}$
& $ R_{_{cond}}^{^{cilin}} = \frac {\ln ( \frac {r_{e}}{r_{i}} )} {2 \pi KL} $ \\
\end{tabular} \\

\begin{tabular}{l r}
$\dot{Q} =\frac{\Delta T}{R_{_{TOT}}} $& $\textrm{Potenza Termica}$ \\
$\dot{q} =\frac{\Delta T}{r_{_{TOT}}}$  &$ \textrm{Flusso Termico} $ \\
$ R = \Big[ \frac {K}{W}\Big ]$ & $ r =\Big [\frac{Km^2}{W} \Big] $ \\
\end{tabular}
\textbf{Generazione interna}: \\
\(\dot{Q_{in}} + \dot{g}V = \dot{Q_{out}} + \frac{dU}{dt} \) \\
Lastra piana con convezione: \\
\(T = -\frac{\dot g}{2k}x^2 + \frac{\dot g}{2k}Lx + T_s\) \\
\(\dot q = -k(-\frac{\dot g}{k}x+\frac{\dot g}{2k}L)\) 

\section*{Convezione}
Legge di Newton per la convezione:\\
\(\dot{Q} = h A (T_s - T_{\infty})\) \\
Il numero di Biot ci indica se possiamo usare l'analisi a parametri concentrati: \\
\( B_{i} = \frac {hL_{c}}{k_{_{materiale}}} \ll 0.01 \quad \vline \quad 
 L_{c} = \frac {V}{A_{_{\text{scambio}}}} \)
\[T(t) = T_{\infty} + {(T_0-T_\infty)}^{e^{- \frac{t}{T}} }\]
\[ \tau = \frac {Mc}{hA_{_{SCAMBIO}}} \quad   = \frac{\rho Vc}{hA} \]
\[t = -  \frac {\rho cV}{hA} \ln \bigg ( \frac {T(t)-T_{\infty}}{T(0)-T_{\infty}}  \bigg )\]
La viscositá cinematica: \(\nu = \frac {\mu}{\rho} \) \\
Perdita di carico: \(\Delta P = f\big(\frac{L}{D}\big)\rho \frac{w^2}{2}\) \\
\[ Re = \frac {w_{\infty} \rho L}{\mu}   = \frac{w_{\infty} L}{\nu} \]
\begin{tabular}{l r}
	Lastra p. & $Re_{cr} = 500000$ \\
	Cil. esterno & $Re_{cr} = 100000$ \\
	Cil. interno & $2300 < \text{instabile} < 4000$ \\
	 & $ 4000 < \text{turbolento}$
\end{tabular}

\[ P_{R} = \frac {c_p\mu}{k} \]
\begin{tiny}differenza con Biot é che il \(k\) usato é quello del fluido non quello del solido\end{tiny}
\[ N_{u} = \frac{\dot{Q_{cnv}}}{\dot{Q_{cnd}}} = \frac {hL}{k} \]
\[T_{film} = \frac{T_s + T_{\infty}}{2}\]

\subsection*{Alette}
Equazione aletta:
\begin{align*}
	-d\dot{Q}_{cond} &= \dot{Q}_{conv} \\
	\frac{d}{dx} \big( \frac{dT}{dx} \big) &= \frac{h\cdot \text{P}_{erimetro}}{k_s\cdot Sez} \cdot (T-T_{\infty})
\end{align*}
Parametro alette: \(m=\sqrt{\frac{h\cdot \text{P}_{erimetro}}{k_s\cdot Sez}}\)
\[L_{c} = L_{fin} + \frac {t}{2} \]
\[\eta_{fin} = \frac{\tanh(m\cdot L)}{m\cdot L}\]
\begin{align*}
\dot{Q}_{tot} &= \dot{Q}_{base} + \dot{Q}_{fin} \\
&= h(A_{base} + A_{fin} \cdot \eta_{fin} ) \cdot (T_s-T_{\infty})
\end{align*}

Coefficiente globale di scambio termico interno
\[ \frac {1}{U_{i}} = \frac {1}{h_{i}} + \frac {D_{i} \ln \Big( \frac {D_{e}}{D_{i}} \Big) } {2k} + \frac {1}{h_{c}} \frac {D_{i}}{D_{e}}  \]
\textbf{Differenza media logaritmica}
\[ \Delta T_{ML} = \frac {\Delta T_{Sn}-\Delta T_{Dx}} { \ln \Big( \frac { \Delta T_{Sn}}{\Delta T_{Dx}} \Big) }   \]









\[ S_{_Q}^{^{OUT}}= - \frac{Q^{^{IN}}}{T_{_{SERB}}} \]

\[L_{_{OUT}}= L_{_{DIL}}- L_{_{DISS}} \]




\[ R_{_{CONV}} =\frac{I}{hA}\]
	




\[ T_{i} = T_{0} - \dot Q \sum _{0}^{i} R \]	





\[L_{_{OUT}}^{^{ISOBARA}}= P \Delta V \]

\[ H = PVU \]

\[ dh = cdt + vdP \]
	
\[ds= c \frac{dT}{T}\]

\section{Costanti}


\[ M_{_{m}}^{^{ARIA}} = 28.9 \quad \Big [ \frac {Kg}{Kmol} \Big ] \ \]

\[ M_{_{m}}^{^{O2}} = 32 \quad \Big [ \frac {Kg}{Kmol} \Big ] \ \]

\[ M_{_{m}}^{^{ELIO}} = 4 \quad \Big [ \frac {Kg}{Kmol} \Big ] \ \]

\[ M_{_{m}}^{^{AZOTO}} = 28 \quad \Big [ \frac {Kg}{Kmol} \Big ] \ \]

\[ M_{_{m}}^{^{ACQUA}} = 18 \quad \Big [ \frac {Kg}{Kmol} \Big ] \ \]

\[ P_{_{AMBIENTE}} = 10135  \quad Pa \]

\[ N_{_{TUBI}} = \Bigg \lceil \frac {\dot m}{\rho \overline w Sez} \Bigg \rceil \]


\section{New section to be retitled}


\[ Nu = cRe^{m} \Pr^{\frac {1}{3}} \]














\section{Compressori}
Trasformazione isoentropica, quindi abbiamo rapporto di compressione
\(Pv^\gamma = \text{cost}\) \\
\[\frac{T_2}{T_1}=\frac{P_2}{P_1}^{\frac{\gamma -1}{\gamma}} \] \\
Rendimento isoentropico \\
\[ \eta_{is}^C = \frac{l_{in}^{is}}{l_{in}} = \frac{h_{2}^{is} - h_1}{h_2 - h_1} =  \frac{T_{2}^{is} - T_1}{T_2 - T_1}\] \\
Lavoro entrante \\
\( \dot L_{in} = \dot m c_p \frac{(T_2 - T_1)}{\eta_{is}^C} \)


\section{Turbine}



\end{multicols}
\end{document}